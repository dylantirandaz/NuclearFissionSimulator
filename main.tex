\documentclass{article}
\usepackage{amsmath} % For math equations

\begin{document}

\title{Mathematical Foundations of Nuclear Fission Simulation}
\author{Dylan Aria Tirandaz}
\date{\today}
\maketitle

\section{Introduction}

This document provides a detailed mathematical explanation of the nuclear fission simulation using Monte Carlo methods, based on computational algorithms used in neutron transport and fission physics.

\section{Fundamental Equations}

The simulation incorporates key equations and concepts essential for understanding neutron interactions and energy transport:

\subsection{Watt Spectrum}

The Watt spectrum describes the distribution of neutron energies emitted during fission, adapted from experimental data and nuclear physics principles:

\begin{equation}
f(E) = a \sinh\left(\sqrt{bE}\right) e^{-aE - bE}
\end{equation}

Here, \( E \) represents neutron energy, and parameters \( a \) and \( b \) are specific to each fissile isotope, influencing the energy distribution of emitted neutrons.

\subsection{Cross-section Interpolation}

Neutron cross-sections for each isotope are crucial for determining interaction probabilities. These cross-sections are interpolated using cubic splines based on experimental data:

\begin{equation}
\sigma(E) = \text{interp1d}(\text{energy\_data}, \text{cross\_section\_data}, \text{kind}='cubic')
\end{equation}

This interpolation method ensures accurate estimation of neutron interaction probabilities across a spectrum of energies.

\subsection{Neutron Scattering}

Neutrons scatter isotropically according to the Maxwell-Boltzmann distribution, influenced by the material's temperature \( T \):

\begin{equation}
f(v) = \sqrt{\frac{2}{\pi}} \left(\frac{v^3}{kT}\right)^{3/2} e^{-\frac{v^2}{2kT}}
\end{equation}

where \( v \) is the neutron speed, \( k \) is the Boltzmann constant, and \( T \) is the material temperature. This distribution governs the velocities of scattered neutrons based on thermal energy considerations.

\section{Simulation Process}

The Monte Carlo simulation progresses through several computational steps mirroring physical processes within a nuclear reactor:

\begin{enumerate}
    \item **Initialization**: Neutrons are initialized within a defined geometry, each with an initial position, direction, and energy distribution from the Watt spectrum.
    \item **Neutron Transport**: Neutrons move through the material in discrete steps, with their paths determined by the mean free path and random walk principles.
    \item **Interaction Mechanisms**: At each step, neutrons may undergo fission, absorption, or scattering interactions based on calculated probabilities from cross-sections and physical constants.
    \item **Temperature Dynamics**: Energy deposition from neutron interactions contributes to the material's temperature evolution, governed by heat capacity and cooling rates.
    \item **Criticality Analysis**: The simulation monitors neutron flux and reaction rates over time to assess criticality conditions, indicating sustained fission reactions.
\end{enumerate}

\section{Conclusion}

This document has outlined the foundational mathematical principles and computational procedures employed in the simulation of nuclear fission using Monte Carlo methods.

\end{document}
